\chapter{Conclusions} \label{Conclusion}
The detection of outliers in indicates something scientifically interesting or significant. There are various algorithms for outlier detection in datasets. Two of them are Local Outlier Factor and One Class Support Vector Machine. Both algorithms have been used to detect outliers in a credit card fraud dataset. These algorithms require labeled normal data, which is essential in semi-supervised learning algorithms. The algorithms assigned a binary label to the observations, classifying themes either normal or fraudulent. Local Outlier Factor and One Class Support Vector Machine have different ways to identify outliers. They use \em{density} and \em{distance} of the observations respectively.

This research proposes a comparison of these two algorithms for the outlier detection process. Both algorithms reach different percentages of performance, and the One Class Support Vector Machine performs better than Local Outlier Factor in the given dataset. Comparison of these algorithms has been done by using the results of a confusion matrix, precisely, the True Positive Rate and False Positive Rate. The algorithm that has a higher True Positive Rate and a False Positive Rate is the better one.

To test the algorithm, a range of parameter values have been used. The Local Outlier Factor True Positive Rate is around 90$\%$, while the False Positive Rate changes from 88.7$\%$ to 79.7$\%$ and the Area Under the Curve is from 52.2$\%$ to 44.7$\%$. The One Class Support Vector Machine algorithm shows different behavior, with True Positive Rates between 98.8$\%$ and 89.4$\%$, and False Positive Rates is between 21.6$\%$ and, while its 10.4$\%$ and the Area Under the Curve is from 93$\%$ to 94$\%$. To conclude, the algorithms have shown different True Positive Rates, False Positive Rates and areas of Receiver Operator Characteristic. One Class Support Vector Machine displays a better performance on detecting outliers.

The current results are obtained by applying the Local Outlier Factor algorithm and One Class Support Vector Machine in some transactions that selected randomly from the dataset. Therefore the behavior of the algorithms may change with different transactions and parameters. Further work could be done by applying the algorithms on the whole dataset which will give more precise results. Applying these algorithms in different datasets to see if the results convergence or divergence will give a better understanding of algorithms behavior. Adding more Machine Learning outlier detection algorithms will help to know which one is better to apply in Big Data.