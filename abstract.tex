\chapter*{Abstract}
\addcontentsline{toc}{chapter}{Abstract}
Detecting outliers in datasets is crucial in many domains. While in many cases outliers are considered a nuisance in the dataset, other times they actually contain important information. There are various Machine Learning methods which have been developed to detect outliers, including Local Outlier Factor and One Class Support Vector Machine.

The purpose of this research is to compare Local Outlier Factor and One Class Support Vector Machines models in order to find the better performing algorithm in a highly unbalanced dataset. The dataset was taken from the Kaggle data science repository. Specifically, a fraudulent credit card dataset was used to evaluate the models.

Evaluation of these models is based on True Positive Rate and False Positive Rate. Comparing results of the algorithms shows that the One Class Support Vector Machine model performs better than the Local Outlier Factor model in the given dataset.
\begin{comment}
Detecting outliers in datasets is valuable in many domains. Identifying the unusual behavior in datasets provide information that is useful. There are various statistical methods used to detect outliers. However, they assume that the data has univariate distribution, which is not generally the case.
There are also various Machine Learning methods which have been developed to detect outliers. Two of the best performing methods are Local Outlier Factor and One Class Support Vector Machine. The Local Outlier Factor model identifies the outliers using \textit{density} of the observation, unlike the One Class Support Vector Machines model which uses the \textit{distance} between observations to find the outliers.

The purpose of this research is to compare Local Outlier Factor and One Class Support Vector Machines models in order to find the better performing algorithm in a highly unbalanced dataset. The dataset was taken from the Kaggle data science repository. Specifically, a fraudulent credit card dataset was used to evaluate the models. The focus of this research will be on these models of semi-supervised techniques, which assume that some of the datasets must be labeled as normal observations.

Evaluation of these models is based on True Positive Rate and False Positive Rate. These rates depend on the model's parameters, thus the experiment has used a range of values for them. Optimal parameters were then chosen to compare the results. Comparing results of the algorithms shows that the One Class Support Vector Machines model performs better than the Local Outlier Factor model in the given dataset.
\end{comment}



\begin{RLtext}
\huge{الملخص}
\end{RLtext}

\begin{RLtext}


تتأثر نتائج التحليل الإحصائي بشكل كبير بوجود قيم شاذة أو متطرفة في البيانات، و بالتالي تظهر أهمية تحديد هذه القيم. حيث يعتبر تعليم الآلة من أفضل النماذج لتحديد هذه القيم. يناقش هذا البحث نموذجين من نماذج تعليم الآلة: أولا نموذج العوامل المحلية الشاذة. ثانيا نموذج  آلة الدعم الموجه ذات الفئة الواحدة. حيث تتم مقارنة نتائج النموذجين من خلال تحديد المعدل الإيجابي الصحيح و المعدل الإيجابي الخاطئ; لمعرفة أي النموذجين أفضل لتحديد القيم الشاذة.
\end{RLtext}

\vfill
\section*{Declaration}
I, the undersigned, hereby declare that the work contained in this research project is my original work and that any work done by others or by myself previously has been acknowledged and referenced accordingly.

% Scan your signature into a small picture called 'signature.png' and insert it
% above your name and the date:
\includegraphics[height=2cm]{images/1} \newline \hrule
% Your name must be in English Capitalisation with no comma, and the Family name comes last. 
% Do note the date below. It is called the "deadline".
Reem Omer Mohammed Elmahdi, 18 May 2017