\section{Mathematical Properties}
Consider $\mathcal{D}$ to be a set of observations, such that $\mathcal{D} = X \biguplus Y$, $\mathcal{D}$ is the set of two sets $X$ and $Y$ with no shared elements between them; $X$ is the set of normal observations, while $Y$ is the set of outliers; Denote $\epsilon$ as threshold; $Fe$ is the feature of a function $F$ of the an observation; Any function $F$ uses feature $Fe$ must satisfy the following properties:

\begin{def}\textbf{(Outlier Existence)} If there are at least two clusters then we should check for outliers:
\begin{align}
degree(D) \geq 2
\end{align}
\end{def}

\begin{def}\textbf{(Normality)} To check whether a certain observation is normal:
\begin{align}
x \in D \text{ is normal observation if } x \in X | d(x) \geq \epsilon
\end{align}
\end{def}

\begin{def}\textbf{(Outlier)} To check whether a certain observation is an outlier:
\begin{align}
y \in D \text{ is an outlier if } y \in Y | d(y) < \epsilon
\end{align}
\end{def}

\section{Motivation}
The motivation of this research was to compare two algorithms of outlier detection. Precisely, One Class Support Vector Machine and Local Outlier Factor, bringing together two advantages:
\begin{itemize}
\item Ability to know the algorithm with the higher efficiency.
\item Ability to know the algorithm with the higher effectiveness.
\end{itemize}

\section{Objective}
In order to achieve the above goals, the following objectives must be satisfied:
\begin{itemize}
\item Simplicity of structure.
\item Functional Completeness.
\end{itemize}

