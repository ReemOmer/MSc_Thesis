\chapter{Conclusions} \label{Conclusion}
There are various algorithms for outlier detection in datasets. Two of them are Local Outlier Factor and One Class Support Vector Machine. Both algorithms have been used to detect outliers in a credit card fraud dataset. These algorithms require labeled normal data, which is essential in semi-supervised learning algorithms. The algorithms assigned a binary label to the observations, where 1 refers to normal observation and -1 refers to an outlier. Local Outlier Factor and One Class Support Vector Machine have different ways to identify outliers. They use \em{density} and \em{distance} of the observations respectively.

This research proposes a comparison of these two algorithms for the outlier detection process. Both algorithms reach different percentages of performance. But One Class Support Vector Machine performs better than Local Outlier Factor in the given dataset.

Comparison of these algorithms has been done by using the results of a confusion matrix, precisely, True Positive Rate, and False Positive Rate. The algorithm that has a higher True Positive Rate and a False Positive Rate is the better one.

The Local Outlier Factor True Positive Rate is from 80.4$\%$ to 70.1$\%$, while the False Positive Rate is from 90.1$\%$ to 90.2$\%$ and the Area Under the Curve is from 35$\%$ to 51$\%$. The One Class Support Vector Machine algorithm shows different behaviour, the True Positive Rate is from 98.1$\%$ to 97.8$\%$, False Positive Rate is from 21.6$\%$ to 14.6$\%$ and the Area Under the Curve is from 93$\%$ to 94$\%$. To conclude, the algorithms have shown different Positive Positive Rate, False Positive Rates and areas of Receiver Operator Characteristic. One Class Support Vector Machine displays a better performance on detecting outliers.

The current results are obtained by applying the Local Outlier Factor algorithm and One Class Support Vector Machine in some transactions that selected randomly from the dataset. Therefore the behaviour of the algorithms may change with different transactions. Further work could be done by applying the algorithms on the whole dataset which will give more precise results. Applying these algorithms in different datasets to see if the results convergence or divergence will give a better understanding of algorithms. Adding more machine learning outlier detection algorithms will help to know which one is better to apply in Big Data.  









